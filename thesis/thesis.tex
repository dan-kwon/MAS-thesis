\documentclass [MS] {uclathes}

%%%%%%%%%%%%%%%%%%%%%%%%%%%%%%%%%%%%%%%%%%%%%%%%%%%%%%%%%%%%%%%%%%%%%%%%
%                                                                      %
%                          PRELIMINARY PAGES                           %
%                                                                      %
%%%%%%%%%%%%%%%%%%%%%%%%%%%%%%%%%%%%%%%%%%%%%%%%%%%%%%%%%%%%%%%%%%%%%%%%

\title          {Augmenting MRI Classified with Synthetic Images \\
                Created via Generative Models}
\author         {Daniel Kwon}
\department     {Statistics}
\degreeyear     {2024}

%%%%%%%%%%%%%%%%%%%%%%%%%%%%%%%%%%%%%%%%%%%%%%%%%%%%%%%%%%%%%%%%%%%%%%%%

\chair          {Yingnian\ Wu}
\member         {Frederic R Paik\ Schoenberg}

%%%%%%%%%%%%%%%%%%%%%%%%%%%%%%%%%%%%%%%%%%%%%%%%%%%%%%%%%%%%%%%%%%%%%%%%

\abstract       {WIP - Check back later}

%%%%%%%%%%%%%%%%%%%%%%%%%%%%%%%%%%%%%%%%%%%%%%%%%%%%%%%%%%%%%%%%%%%%%%%%

\usepackage{subcaption} 
\usepackage{graphicx}
\usepackage{amsmath}
\usepackage{amsfonts}
\usepackage{array, makecell}
\usepackage{multirow}
\usepackage{float}
\renewcommand\cellset{\renewcommand\arraystretch{0.8}%
\setlength\extrarowheight{0pt}}

%%%%%%%%%%%%%%%%%%%%%%%%%%%%%%%%%%%%%%%%%%%%%%%%%%%%%%%%%%%%%%%%%%%%%%%%

\begin {document}

\makeintropages

%%%%%%%%%%%%%%%%%%%%%%%%%%%%%%%%%%%%%%%%%%%%%%%%%%%%%%%%%%%%%%%%%%%%%%%%

\chapter{Introduction}
With the performance of state-of-the-art generative models improving rapidly, synthetic images can be produced with ease 
using simple text or image prompts. Image generation models such as Dall-E or Stable Diffusion are able to create images 
with sufficient quality that the margin of difference between real and synthetic images is becoming increasingly thin. 
As the level of effort in generating synthetic images decreases and the quality of these images increases, the potential 
to use synthetic images as a means to augment image datasets becomes increasingly viable---especially in fields where 
gathering images is constrained by costs or other resources.

In the field of medical imaging, where image datasets often require specialized equipment and subject matter experts to 
capture and label images, gathering enough data to sufficiently train a classification model can be both expensive and 
time-consuming. For example, with the cost of magnetic resonance imaging (MRI) ranging from \$1,600 to \$8,400 in the 
United States, a dataset consisting of a few hundred images can exceed \$1 million; image classification models often 
require thousands of photos. 

One way computer vision models have historically attempted to remedy insufficient training datasets has been to employ 
traditional image augmentation techniques, which involve applying transformations such as flipping or blurring an 
original image to produce additional images for training. However, such transformations must be applied carefully in 
order to not lose the fidelity needed to make accurate diagnostic predictions. Transformations that alter the nature of 
the images can potentially lead to a degradation in model performance.

Augmenting image datasets with high quality synthetic images may allow for significant cost-saving while maintaining 
model performance, and previous research has found that synthetic images may play a complimentary role to image 
augmentation. The goal of this paper is to train image classification models on a MRI dataset to identify the presence 
of varying levels of dementia and investigate the effect of different image augmentation policies on model performance 
as well as compare their performance against an augmentation policy that generates synthetic images.

\chapter{Dataset}

\section{Alzheimer MRI Preprocessed Dataset}
For the purposes of this paper, we use publicly available MRI images of patients with varying levels of dementia, 
labeled as \textit{not demented}, \textit{very mildly demented}, \textit{mildly demented}, and \textit{moderately 
demented}. These images were downloaded via the datasets module provided and maintained by Hugging Face. All images are 
in black and white and have been pre-processed to 128x128 resolution.

\begin{figure}[H]
    \centering
    \subfloat[Not Demented]{\includegraphics[width=0.225\textwidth]{figures/not-demented-example.png}\label{fig:f1}}
    \hfill
    \subfloat[Very Mildly Demented]{\includegraphics[width=0.225\textwidth]{figures/very-mild-demented-example.png}\label{fig:f2}}
    \hfill
    \subfloat[Mildly Demented]{\includegraphics[width=0.225\textwidth]{figures/mild-demented-example.png}\label{fig:f3}}
    \hfill
    \subfloat[Moderately Demented]{\includegraphics[width=0.225\textwidth]{figures/moderate-demented-example.png}\label{fig:f4}}
    \caption{Examples of real MRI images}
\end{figure}

\begin{table}[H]
    \centering
    \begin{tabular}{|>{\centering\arraybackslash}p{0.2\linewidth}|>{\centering\arraybackslash}p{0.2\linewidth}|>{\centering\arraybackslash}p{0.2\linewidth}|>{\centering\arraybackslash}p{0.2\linewidth}|} \hline 
        Not Demented & Very Mildly Demented & Mildly Demented & Moderately Demented\\ \hline 
        2566 & 1781 & 724 & 49\\ \hline
    \end{tabular}
    \caption{Count of each class in training data}
    \label{tab:my_label}
\end{table}

\section{Synthetic Images}
This paper utilizes OpenAI's Dall-E-2 to produce synthetic images. To generate synthetic images, an original image is 
supplied as a source and the generative models are prompted to creating a similar image.

In order to generate images using Dall-E-2 we utilize OpenAI's image variation API endpoint, which returns a variation 
of a given image. 

\begin{figure}[H]
    \centering
    \subfloat[Real Image]{\includegraphics[width=0.25\textwidth]{figures/ModerateDemented_9.png}\label{fig:f1}}
    \hspace{0.1\textwidth}
    \subfloat[Synthetic Image]{\includegraphics[width=0.25\textwidth]{figures/ModerateDemented_9_generated30.png}\label{fig:f2}}
    \caption{Results of Dall-E-2's image variation generation}
\end{figure}

Dall-E-2 reliably generates images that are similar to the real MRIs that are provided. Given that dementia is often 
physically tied to brain atrophy in certain areas, the synthetic images produced by Dall-E-2 generally reproduce the 
ridges, folds, and cavities of the source images as well.

\chapter{Exploration using Grad-CAM}
In order to better understand what parts of an image classification models deem to be more "important" in regards to 
predicting the presence of dementia, we employ Gradient-weighted Class Activation Mapping, or Grad-CAM, as a way to 
represent what a convolutional neural network "sees" when classifying both real and synthetic images. Grad-CAM is a 
technique that maps the gradients of a final convolutional layer in regards to a specific class prediction to product a 
heat map. The result is a visually intuitive way of illustrating which portions of an image contribute most to a given 
classification [source here].

\section{Grad-CAM details}
\begin{enumerate}
    \item First, calculate the gradients of the model’s output in the final convolutional layer, with respect to the 
    feature maps. Assuming \(y_{c}\) is the score for class \(c\) (before softmax) and \(\delta A^{k}\) is the feature 
    map activation of the \(k\)th layer, compute \(\frac{\delta y_{c}}{\delta A^{k}}\). 
    
    \item Next, calculate the global average pooling of the feature map. Global average pooling is a pooling operation 
    designed to generate one feature map for each corresponding category of the classification task in the last 
    [convolutional] layer and then take the average of each feature map [source here].
    
    \[\alpha_{k}^{c} = \frac{1}{Z}\sum_{i}\sum_{j}\frac{\delta y_{c}}{\delta A_{ij}^{k}}\]

    \(Z\) here represents spatial dimensions of the feature map---in this case, its height and width. By dividing by 
    \(Z\), we obtain the average of gradients over all spatial locations. \(\frac{\delta y_{c}}{\delta A_{ij}^{k}}\) is 
    the gradient for class c with respect to the activation at \(A_{ij}^{k}\), or spatial location (i,j) on activation 
    layer \(k\).
    
    \item Lastly, we take the resulting importance weights in \(a_{k}^{c}\) and combine them with \(A^{k}\) to calculate 
    a weighted combination of all forward activation maps. We then apply the ReLU activation function to the resulting 
    combination in order to take the positive values only. This ensures that we are only looking parts of the image that 
    are positively correlated with a given class.

    \[L_{Grad-CAM}^{c} = ReLU(\sum_{k}\alpha_{k}^{c}A^{k})\]

    Notice that \(L_{Grad-CAM}^{c}\) will have the dimensions of the final convolutional layer and therefore is likely 
    to be smaller than the original input image. In that case, we simply upsample the result to the dimensions of our 
    original image in order to overlay the two.
\end{enumerate}

\section{Grad-CAM results}
When Grad-CAM is 

\chapter{Background on Training Deep Learning Models}

\subsection{Overview of Neural Networks}
While deep learning architectures can span many layers and can incorporate many different mechanisms, at its core all 
deep learning models are neural networks. Training neural networks comprise of a few key steps. 

First, training data is input into a neural network and passed through as-is in what is known as a "forward pass." In a 
fully connected neural network, every individual neuron in a layer of the neural network is connected to each neuron in 
the preceding layer 

\begin{figure}[H]
    \centering
    \includegraphics[width=0.75\linewidth]{figures/NN_diagram.jpg}
    \caption{An illustration of a  typical fully-connected neural network}
    \label{fig:enter-label}
\end{figure}

Each of these connections has a weight that stores how strong of a connection each preceding node has to the current 
node. A bias term is also present, which represents whether a neuron is general activated or not. Often, an initialized 
neural network will comprise of randomized weights and biases whereas a pre-trained model will have the weights and 
biases already defined from previous training. When training a neural network, the weights and biases are what are being 
adjusted in order to minimize loss. The formulation of a single neuron with n nodes in the previous layer is below:

\[\sigma(w_0a_0 + w_0a_0 + w_0a_0 + ... + + w_{n-1}a_{n-1} + b)\]

After the forward pass, backpropagation occurs in which the gradient (the vector of partial derivatives) of the loss 
function with respect to each weight and bias is calculated via chain rule. Each gradient value represents the magnitude 
and direction of the change in our loss function given a change to that particular weight or bias. Because 
backpropagation uses the chain rule to propagate the loss backward from the output layer to the input layer, the process 
of finding the gradient of a loss function remains the same no matter how many layers are in the network or how many 
neurons are in each layer.

Once the gradient is calculated, the gradient descent algorithm is used to update the weights and biases in order to 
minimize loss. However, applying the gradient descent on an entire training dataset in a single batch is computationally 
expensive. Instead, gradient descent is often applied to subsets of the training data, called mini-batches. Each 
training iteration, or epoch, will then take a mini-batch to apply the forward and backward passes on to to updates its 
weights and biases. The result is an accurate approximation of the gradient of the loss function while significantly 
decreasing computational expenses [source].

\subsection{Loss Function}
The loss function used throughout this paper when measuring model performance is cross entropy loss. Cross entropy loss 
is defined as:

\[L = -\sum_{c=1}^{C}y_{c}log(p_{c})\]

where \(C\) = the number of classes, \(y_{c}\) is the true label for class, and \(p_{c}\) is the predicted probability 
for class \(c\). Cross entropy loss therefore compares the predicted probabilities with the actual labels and penalizes 
more when the predicted probabilities for a correct class is low [source].

Cross entropy loss is better suited for measuring the performance of classification models when multiple classes are 
involved than traditional measures of error, such as the sum of residuals, as it is more sensitive to predictions that 
are "more" incorrect. In an image classification problem, a model may mislabel a given image, but the cross entropy loss 
will be lower if the predicted probability for the correct class is higher, even if the model did not ultimately end up 
labeling correctly. Contrast that with the sum of residuals, which only views predictions as completely correct or 
completely incorrect and fails distinguish between the degrees of how right or wrong a prediction can be [source].

\begin{figure}[H]
    \centering
    \includegraphics[width=1\linewidth]{figures/CE vs Square Residual.jpg}
    \caption{Illustration of Cross Entropy Loss vs Squared Residual}
    \label{fig:enter-label}
\end{figure}

Take for example, when the true label is 1 and the predicted class weight \(y_{c}\) is 0.0001---making our model 
prediction very bad. The squared residual would be \((1-0.0001)^{2} = 0.9998\) while our cross entropy loss would be 
more punitive at \(-log(0.0001) = 4\). Cross entropy loss also has a much larger gradient for very bad predictions, 
allowing our model to more quickly learn to avoid bad predictions.

\subsection{Stochastic Optimization}
All models mentioned in this paper are trained using the Adaptive Mom

\newpage
\section{Overview of Model Architectures}
We test two different architectures for our image classification models---a basic Convolutional Neural Network (CNN) and 
a Vision Transformer (ViT)---in order to gain an understanding as to how both convolution-based and attention-based 
architectures react to the introduction to synthetic data in training. While the details of either architectures are 
outside the scope of this paper, a brief over of both are provided below.

\subsection{Convolutional Neural Networks}

\begin{figure}[h]
    \centering
    \includegraphics[width=0.9\linewidth]{figures/CNN-diagram.jpg}
    \caption{Typical CNN Architecture}
    \label{fig:synthetic-mix}
\end{figure}

A CNN a deep learning architecture that is comprised of convolutional layers---which abstract an image to a feature map, 
pooling layers---which reduce the dimensions of data by combining the outputs of adjacent layers via downsampling, and 
fully connected layers---which are neural networks that take the final activations from the convolutional and pooling 
layers to generate class weights.

For the purposes of this paper, we utilize a CNN with 2 convolutional layers, each paired with a ReLU activation 
function and a max pooling layer. After all feature extraction layers are complete, the final activation layer is fed 
into a fully connected neural network with 2 hidden layers and an output layer that predicts probabilities weights for 
each of our four classes.

\subsection{Vision Transformers}
WIP

\newpage
\section{Methodology}
\subsection{Traditional Image Augmentation Policies}
\subsection{Synthetic Image Generation Policy}
Given a desired proportion of original training images to utilize (\textit{X}) and the desired proportion of synthetic 
images to augment the training data (\textit{Y}), we combine both synthetic and original images to train our image 
classification model. 

\begin{figure} [h]
    \centering
    \includegraphics[width=1\linewidth]{figures/image_generation_policy.jpg}
    \caption{Enter Caption}
    \label{fig:enter-label}
\end{figure}

The figure above illustrates one iteration of our image generation policy in order to generate our mixed training set 
(i.e., containing both original and synthetic images)
\begin{enumerate}
    \item First, we take a random stratified sample of our original images
    \item The stratified sample is sent to Dall-E 2 via OpenAI's API in order to generate an image variation
    \item Image transformations are applied to our entire mixed training set (if applicable)
    \item Mixed training set, post transformations/augmentation, is used to train the image classification model
    \item Steps 1-4 are repeated \textit{N} times in order to account for the stochastic nature of Dall-E's responses as 
    well as to bootstrap a distribution of performance--measured by cross entropy loss
\end{enumerate}


\newpage
\section{Results}

\subsection{CNN Results}
When applied to a basic convolutional neural network, augmentation via synthetic images provides the most benefit when 
the dataset of original images is limited. When training our CNN on a 30\% stratified sample, the median validation loss 
from 100 simulations was 0.219863, compared to a loss of 0.176023 when training on the full dataset---resulting in an 
increase in our loss by approximately 25\%.

However, model performance improves when augmenting the 30\% stratified training sample with synthetic images so that 
the resulting training set contains the same number of images as the full training data set. The resulting model trained 
on the mix of 30\% real and 70\% synthetic images has a cross-entropy loss of 0.174202---a decrease of over 20\% 
compared to the model trained on only the 30\% of real training images and matching the performance of model trained on 
the entirety of the training images.

\begin{figure}[H]
    \centering
    \includegraphics[width=0.7\linewidth]{figures/cnn_minimal30_result.png}
    \caption{Cross Entropy Loss - Model trained on 30\% Real, 70\% synthetic}
    \label{fig:enter-label}
\end{figure}

Moreover, CNNs trained on synthetic data outperformed CNNs trained on 30\% of available training images and paired with 
traditional image augmentation policies. This is likely due to the fact that the image augmentation policies may alter 
the nature of the image enough to degrade performance of our classification model. This is not to say that traditional 
image augmentation methods do not have a role in medical imaging; only that traditional image segmentation policies, 
when misapplied, can degrade model performance significantly and therefore likely require subject matter experts to 
evaluate whether image augmentation is necessary. In fact, other studies have found that image augmentation generally 
improves the performance of image classification models used for brain imaging. 

\begin{figure}[H]
    \centering
    \includegraphics[width=0.7\linewidth]{figures/image_augmentation_comparison.png}
    \caption{Comparison of resulting cross entropy loss across models using different image augmentation policies, all 
    trained on 30\% of available images}
    \label{fig:enter-label}
\end{figure}
However, not all training datasets benefit from synthetic image augmentation. CNNs trained on 40-80\% of all available 
training data resulted in virtually no improvement, with the 80\% model resulting in a degradation in loss. This alludes 
to the possibility that synthetic data augmentation may be more beneficial as original training data becomes more 
constrained.

\chapter{Conclusion}

\chapter{Additional Considerations}

\end {document}